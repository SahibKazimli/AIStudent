\documentclass{beamer}
\usepackage{amsmath}
\usepackage{hyperref}
\usepackage{graphicx}
\usepackage{tikz}
\addtobeamertemplate{frametitle}{}{%
    \begin{tikzpicture}[remember picture,overlay]
        \node[anchor=north east, xshift=-0.5cm, yshift=-0.1cm] at (current page.north east) {
            \includegraphics[width=1.5cm]{imgs/kth-logo.png}
        };
    \end{tikzpicture}
}

\title{Lecture: Deep Learning Fundamentals}
\author{KTH AI Student}
\date{\today}

\begin{document}

\frame{\titlepage}

\section{Introduction}
\begin{frame}{What we will cover today}
    \begin{itemize}
        \item History of Deep Learning.
        \item Fundamentals of a Perceptron.
        \item Understanding and Implementing Gradient Descent.
        \item Introduction to Autograd and Backpropagation.
        \item Training a Multilayer Perceptron (MLP).
        \item Deep Dive into Convolutional Neural Networks (CNNs).
    \end{itemize}
\end{frame}

\section{History of Deep Learning}
\begin{frame}{Early Beginnings}
    \begin{itemize}
        \item 1943: McCulloch and Pitts propose the first mathematical model of a neuron.
        \item 1958: Rosenblatt introduces the Perceptron, an early neural network.
    \end{itemize}
\end{frame}

\begin{frame}{First AI Winter}
    \begin{itemize}
        \item 1970s: AI research faces criticism and reduced funding.
        \item Limitations of Perceptrons highlighted by Minsky and Papert.
    \end{itemize}
\end{frame}

\begin{frame}{Revival and Second AI Winter}
    \begin{itemize}
        \item 1980s: Backpropagation algorithm popularized by Rumelhart, Hinton, and Williams.
        \item 1990s: AI research faces another decline due to unmet expectations.
    \end{itemize}
\end{frame}

\begin{frame}{Milestones in AI}
    \begin{itemize}
        \item 1997: IBM's Deep Blue defeats world chess champion Garry Kasparov.
        \item 2006: Hinton and colleagues introduce deep belief networks, sparking renewed interest in deep learning.
    \end{itemize}
\end{frame}

\begin{frame}{Modern Deep Learning Era}
    \begin{itemize}
        \item 2012: AlexNet wins ImageNet competition, demonstrating the power of deep convolutional networks.
        \item 2014: GANs introduced by Goodfellow et al., enabling generative models.
        \item 2017: Transformers introduced by Vaswani et al., revolutionizing NLP.
        \item 2023: Large language models (LLMs) like GPT-3 and BERT achieve human-level performance on various tasks.
    \end{itemize}
\end{frame}

\section{Linear Regression}
\begin{frame}{What is Linear Regression?}
    \begin{itemize}
        \item Linear regression is a simple model used to predict continuous values.
        \item It assumes a linear relationship between input features and output.
        \item The model is represented as:
        \[
        y = wx + b
        \]
        \item Where \(y\) is the output, \(x\) is the input, \(w\) is the weight, and \(b\) is the bias.
    \end{itemize}
\end{frame}

\section{Perceptron and Gradient Descent}
\begin{frame}{What is a Perceptron?}
    \begin{itemize}
        \item A Perceptron is a simple neural network unit.
        \item It takes input features, applies weights, and produces an output.
        \item The output is passed through an activation function.
        \item The equation for a simple linear model is:
        \[
        y = wx + b
        \]
        \item Then we apply an activation function to \(y\).
        \item Example: Sigmoid, ReLU, Tanh.
    \end{itemize}
\end{frame}


\begin{frame}{Perceptron Example}
    \begin{itemize}
        \item Consider a Perceptron with one input feature.
        \item Weights: \(w = 2\), bias: \(b = 1\).
        \item Input: \(x = 3\).
        \item Output: \(y = 2 \times 3 + 1 = 7\).
        \item Apply ReLU activation: \(f(y) = \max(0, y) = 7\).
    \end{itemize}
\end{frame}


\begin{frame}{Perceptron}
    \begin{figure}
        \centering
\includegraphics[width=0.7\textwidth]{perceptron.png}
        \caption{Perceptron}
    \end{figure}
\end{frame}

\begin{frame}{Linear Regression Assumptions}
    \begin{itemize}
        \item Linearity: The relationship between dependent and independent variables is linear.
        \item Independence: Observations are independent of each other.
        \item Homoscedasticity: Constant variance of residuals.
        \item Normality: Residuals should be normally distributed.
    \end{itemize}
\end{frame}

\begin{frame}{How can we learn non-linear relationships?}
    \begin{itemize}
        \item We can use \textbf{Multilayer Perceptrons (MLPs)}.
        \item MLPs consist of multiple layers of neurons.
        \item Hidden layers introduce non-linearity.
        \item Activation functions like ReLU are used to introduce non-linearity.
    \end{itemize}
\end{frame}


\section{Multilayer Perceptron (MLP)}
\begin{frame}{What is an MLP?}
    \begin{itemize}
        \item A \textbf{Multilayer Perceptron (MLP)} consists of multiple layers of neurons.
        \item It includes an \textbf{input layer, hidden layers, and an output layer}.
        \item Uses \textbf{activation functions} to introduce non-linearity.
    \end{itemize}
\end{frame}

\begin{frame}{Common Activation Functions}
    \begin{itemize}
        \item \textbf{Sigmoid:} Used in binary classification problems.
        \[
        \sigma(x) = \frac{1}{1 + e^{-x}}
        \]
        \item \textbf{ReLU (Rectified Linear Unit):}
        \[
        f(x) = \max(0, x)
        \]
        \item \textbf{Tanh:} Scales input to range \([-1, 1]\).
        \[
        \tanh(x) = \frac{e^x - e^{-x}}{e^x + e^{-x}}
        \]
    \end{itemize}
\end{frame}

\begin{frame}
    \begin{figure}
        \centering
        \includegraphics[width=0.7\textwidth]{mlp.png}
        \caption{Multilayer Perceptron}
    \end{figure}
\end{frame}

\begin{frame}{MLP Architecture}
    \begin{itemize}
        \item Input layer: Receives input features.
        \item Hidden layers: Perform computations and transformations.
        \item Output layer: Produces final predictions.
        \item Example: MLP for digit classification.
    \end{itemize}
\end{frame}

\begin{frame}{Training an MLP}
    \begin{itemize}
        \item Initialize weights and biases.
        \item Forward pass: Compute predictions.
        \item Compute loss: Measure prediction error.
        \item Backward pass: Compute gradients.
        \item Update weights: Apply gradient descent.
        \item Repeat until convergence.
    \end{itemize}
\end{frame}

\begin{frame}{Loss Functions}
    \begin{itemize}
        \item \textbf{Mean Squared Error (MSE):} Used for regression problems.
        \[
        \text{MSE} = \frac{1}{N} \sum (y_i - \hat{y}_i)^2
        \]
        \item \textbf{Cross-Entropy Loss:} Used for classification problems.
        \[
        \text{Cross-Entropy} = -\frac{1}{N} \sum y_i \log(\hat{y}_i)
        \]
    \end{itemize}
\end{frame}

\begin{frame}{Gradient Descent Algorithm}
    \begin{itemize}
        \item An optimization technique used to minimize the loss function.
        \item The parameter updates are:
        \[
        w := w - \alpha \frac{\partial \text{MSE}}{\partial w}
        \]
        \[
        b := b - \alpha \frac{\partial \text{MSE}}{\partial b}
        \]
        \item \(\alpha\) is the learning rate, which controls step size.
    \end{itemize}
\end{frame}

\begin{frame}{Gradient Descent Algorithm}
    \begin{figure}
        \centering
        \includegraphics[width=0.7\textwidth]{gradient.png}
        \caption{Gradient Descent}
    \end{figure}
\end{frame}

\begin{frame}{Challenges in Gradient Descent}
    \begin{itemize}
        \item Choosing the correct learning rate is critical.
        \item Too small \(\alpha\) results in slow convergence.
        \item Too large \(\alpha\) may cause divergence.
        \item Local minima vs. global minima.
    \end{itemize}
\end{frame}

\begin{frame}{Variants of Gradient Descent}
    \begin{itemize}
        \item \textbf{Batch Gradient Descent:} Uses the entire dataset to compute gradients.
        \item \textbf{Stochastic Gradient Descent (SGD):} Uses one sample at a time.
        \item \textbf{Mini-batch Gradient Descent:} Uses a subset of the dataset.
        \item \textbf{Momentum:} Accelerates convergence by considering past gradients.
        \item \textbf{Adam:} Adaptive learning rate method combining momentum and RMSProp.
    \end{itemize}
\end{frame}

\begin{frame}{Mathematical Derivation of Gradient Descent}
    \begin{itemize}
        \item The gradient of MSE with respect to \(w\) is:
        \[
        \frac{\partial \text{MSE}}{\partial w} = -\frac{2}{N} \sum x_i (y_i - (wx_i + b))
        \]
        \item The gradient of MSE with respect to \(b\) is:
        \[
        \frac{\partial \text{MSE}}{\partial b} = -\frac{2}{N} \sum (y_i - (wx_i + b))
        \]
        \item Update rules:
        \[
        w := w - \alpha \left( -\frac{2}{N} \sum x_i (y_i - (wx_i + b)) \right)
        \]
        \[
        b := b - \alpha \left( -\frac{2}{N} \sum (y_i - (wx_i + b)) \right)
        \]
    \end{itemize}
\end{frame}

\begin{frame}{Backpropagation: The Core of Neural Networks}
    \begin{itemize}
        \item Backpropagation is used to adjust weights to minimize loss.
        \item Steps:
        \begin{enumerate}
            \item Compute forward pass (prediction).
            \item Calculate loss function.
            \item Compute gradients of loss with respect to parameters.
            \item Update weights using gradient descent.
        \end{enumerate}
    \end{itemize}
\end{frame}

\begin{frame}{Chain Rule in Backpropagation}
    \begin{itemize}
        \item The derivative of a composite function:
        \[
        \frac{df}{dx} = \frac{df}{du} \times \frac{du}{dx}
        \]
        \item Used to efficiently propagate gradients in deep networks.
    \end{itemize}
\end{frame}

\begin{frame}{Example of Backpropagation}
    \begin{itemize}
        \item Consider a simple neural network with one hidden layer.
        \item Forward pass:
        \[
        z_1 = w_1 x + b_1
        \]
        \[
        a_1 = \sigma(z_1)
        \]
        \[
        z_2 = w_2 a_1 + b_2
        \]
        \[
        \hat{y} = \sigma(z_2)
        \]
        \item Loss function:
        \[
        L = \frac{1}{2} (\hat{y} - y)^2
        \]
    \end{itemize}
\end{frame}

\begin{frame}{Backpropagation Steps}
    \begin{itemize}
        \item Compute gradients:
        \[
        \frac{\partial L}{\partial \hat{y}} = \hat{y} - y
        \]
        \[
        \frac{\partial \hat{y}}{\partial z_2} = \sigma'(z_2)
        \]
        \[
        \frac{\partial z_2}{\partial w_2} = a_1
        \]
        \item Chain rule:
        \[
        \frac{\partial L}{\partial w_2} = \frac{\partial L}{\partial \hat{y}} \cdot \frac{\partial \hat{y}}{\partial z_2} \cdot \frac{\partial z_2}{\partial w_2}
        \]
        \item Update weights:
        \[
        w_2 := w_2 - \alpha \frac{\partial L}{\partial w_2}
        \]
    \end{itemize}
\end{frame}

\begin{frame}{Regularization Techniques}
    \begin{itemize}
        \item \textbf{L2 Regularization:} Adds penalty for large weights.
        \[
        L = \text{MSE} + \lambda \sum w^2
        \]
        \item \textbf{Dropout:} Randomly drops neurons during training.
        \item \textbf{Early Stopping:} Stops training when validation loss increases.
    \end{itemize}
\end{frame}

\section{Convolutional Neural Networks (CNNs)}
\begin{frame}{What is a CNN?}
    \begin{itemize}
        \item CNNs are specialized for image recognition and processing.
        \item Use \textbf{convolutional layers} to extract important features.
        \item Pooling layers reduce dimensionality while retaining information.
    \end{itemize}
\end{frame}

\begin{frame}{Convolutional Layers}
    \begin{itemize}
        \item Apply filters (kernels) to detect features like edges.
        \item Example: Using a \textbf{3x3 kernel} for edge detection.
        \item Convolution operation:
        \[
        (I * K)(i, j) = \sum_m \sum_n I(i+m, j+n) K(m, n)
        \]
        \item Where \(I\) is the input image and \(K\) is the kernel.
    \end{itemize}
\end{frame}

\begin{frame}{Pooling Layers}
    \begin{itemize}
        \item Reduce the size of feature maps while preserving important features.
        \item \textbf{Max pooling} selects the highest value in a region.
        \item \textbf{Average pooling} computes the average value in a region.
    \end{itemize}
\end{frame}

\begin{frame}{Fully Connected Layer}
    \begin{itemize}
        \item After convolution and pooling, a fully connected layer is used for classification.
        \item Example: Final layer in \textbf{image classification} predicts categories.
    \end{itemize}
\end{frame}

\begin{frame}{CNN Architecture}
    \begin{itemize}
        \item Input layer: Receives image data.
        \item Convolutional layers: Extract features.
        \item Pooling layers: Reduce dimensionality.
        \item Fully connected layers: Perform classification.
        \item Example: LeNet-5 for digit recognition.
    \end{itemize}
\end{frame}

\begin{frame}{Training a CNN}
    \begin{itemize}
        \item Initialize filters and weights.
        \item Forward pass: Compute feature maps and predictions.
        \item Compute loss: Measure prediction error.
        \item Backward pass: Compute gradients.
        \item Update weights: Apply gradient descent.
        \item Repeat until convergence.
    \end{itemize}
\end{frame}

\begin{frame}{CNN}
    \begin{figure}
        \centering
        \includegraphics[width=1\textwidth]{cnn.png}
        \caption{Convolutional Neural Network}
    \end{figure}
\end{frame}

\begin{frame}{Data Augmentation}
    \begin{itemize}
        \item Technique to increase the diversity of training data.
        \item Examples: Rotation, scaling, flipping, and cropping.
        \item Helps prevent overfitting and improves generalization.
    \end{itemize}
\end{frame}

\begin{frame}{Transfer Learning}
    \begin{itemize}
        \item Use pre-trained models on large datasets.
        \item Fine-tune the model on a specific task.
        \item Example: Using VGG16 for image classification.
    \end{itemize}
\end{frame}

\section{Current Architectures}
\begin{frame}{Recurrent Neural Networks (RNNs)}
    \begin{itemize}
        \item Designed for sequential data.
        \item Maintain hidden states to capture temporal dependencies.
        \item Applications: Time series prediction, language modeling.
    \end{itemize}
\end{frame}

\begin{frame}{Long Short-Term Memory (LSTM)}
    \begin{itemize}
        \item A type of RNN designed to handle long-term dependencies.
        \item Uses gates to control the flow of information.
        \item Applications: Speech recognition, text generation.
    \end{itemize}
\end{frame}

\begin{frame}{Generative Adversarial Networks (GANs)}
    \begin{itemize}
        \item Consist of a generator and a discriminator.
        \item Generator creates fake data, discriminator distinguishes real from fake.
        \item Applications: Image generation, data augmentation.
    \end{itemize}
\end{frame}

\begin{frame}{Transformers}
    \begin{itemize}
        \item Use self-attention mechanisms to process sequences in parallel.
        \item Revolutionized NLP tasks.
        \item Applications: Machine translation, text summarization.
    \end{itemize}
\end{frame}

\begin{frame}{Large Language Models (LLMs)}
    \begin{itemize}
        \item Built on transformer architecture.
        \item Trained on vast amounts of text data.
        \item Applications: Chatbots, content generation, question answering.
    \end{itemize}
\end{frame}

\begin{frame}{Resources}
    \begin{figure}
        \centering
        \includegraphics[width=0.7\textwidth]{dl.png}
        \caption{Resources}
    \end{figure}
\end{frame}

\section{Conclusion}
\begin{frame}{Summary}
    \begin{itemize}
        \item \textbf{Linear Regression} is a fundamental prediction method.
        \item \textbf{Gradient Descent} optimizes model parameters.
        \item \textbf{Autograd} automates differentiation for neural networks.
        \item \textbf{MLPs} introduce complexity beyond linear models.
        \item \textbf{CNNs} excel in image-related tasks.
        \item \textbf{RNNs, LSTMs, GANs, Transformers, and LLMs} represent current advancements in deep learning.
    \end{itemize}
\end{frame}

\end{document}
